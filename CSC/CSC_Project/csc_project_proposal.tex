\documentclass[10pt,a4paper]{article}
\usepackage[utf8]{inputenc}

\title{%
  COMSOC: Project Proposal \\
  \small Grzegorz Lisowki, Max Rapp and Haukur J{\'o}nnson and Silvan Hungerb{\"u}hler } 

\usepackage{mathptmx} % "times new roman"
\usepackage{amssymb}
\usepackage{amsmath, amsthm}
\usepackage{amsfonts}
\usepackage{enumitem}
\usepackage{verbatim}
\usepackage{hyperref}
\usepackage{comment}
%\usepackage[margin=1in]{geometry}
\usepackage{float}
\usepackage{bm}

\usepackage[normalem]{ulem}
\date{}

\begin{document}
\maketitle
Given a set of "voters"/readers $N$ a set of news items $A$ and a corresponding profile $R$ we want to fill a set of relevant alternatives $M$. We do this by iterating a resolute voting rule $F:\mathcal{L}(A)^n \rightarrow A$ $m$ times.\\

Questions:\\
- Which axioms should the relevant set $M$ fulfill and to which axioms on the voting rule F do they correspond?
\section*{Introduction}
Our project aims at better understanding of collective recommendation mechanisms in media settings. We want to use formal tools provided by Social Choice Theory to analyze benefits and drawbacks of various possible ways to determine a set of essential news items - eg. newspaper articles - for a group, given each member's individual preferences over the topics - eg. politics, sport, business - instantiated by the items.

News consumers only have so much time and cognitive ressources at their disposal, yet receiving information through news requires both of these. At the same time, it is easy to see the desirability of people - as members in a community or citizens in a state- to be on the same page with respect to topics of high concern to other members. It is, therefore, important to get on understanding of how extract a common core of pieces of information from the dispersed interests and pet topics of individuals while, at the same, time respecting the constraints given by attention span and time.

\begin{enumerate}
\item Introduction: High-Level motivation. Media context
\item Formal Details. Sequential Voting. 
\item Possible avenues: Axiomatic \& Optimization Problem
\end{enumerate}


Material:
Phd Thesis: https://www.mimuw.edu.pl/~ps219737/phdThesis.pdf

\end{document}