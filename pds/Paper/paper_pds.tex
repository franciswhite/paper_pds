\documentclass{article}
\title{Philosophy of Distributional Semantics\\
\large Best Practices for the Application of Computational Methods to Questions of Meaning}
\date{}
\author{Silvan Hungerb{\"u}hler}
\usepackage{enumitem}
\usepackage{verbatim}
\usepackage{hyperref}
\usepackage{comment}
\usepackage[utf8]{inputenc}
\usepackage[english]{babel}
\usepackage{amsmath}

%\usepackage{biblatex}
%\addbibresource{references_wiggy.bib}
\usepackage{csquotes}
\begin{document}
\maketitle
\begin{abstract}
The increase in availability of data and advances in statistical methods has led to the use Machine Learning (ML) technology in an immense variety of societal domains. ML is used to inform decisions as diverse and life changing as whether somebody obtains financial credit, gets admitted to university, is medically treated or left on probation.
The use of this technology is not always unproblematic, as its careless application can cause serious harm, even if such effects may occur unintentionally.
The seriousness of the potential to systematically disadvantage people has led to a recent debate concerning ethical guidelines and best practices for the \emph{general} use of ML.
Little has been said, however, about the proper use of ML techniques in the specific field of Distributional Semantics (DS). Some researchers have called attention to possible dangers of using ML in DS, as it harbours the potential to reinforce harmful biases along dimensions of race and gender.
The present paper contains an overview over the current state of the \emph{general} discussion 
and draws on these existing insights from different domains of ML to provide guidelines and best practices for computational methods in DS.
As the domain of DS differs in important aspects from other domains of ML, 
we assess these difference with respect on the impact they have on the applicability of general purpose ML guidelines to DS.
\end{abstract}
\section{Introduction}
\paragraph{Motivation}
1-2 paragraphs, talk about application, overal field
\paragraph{Contribution}
largest part of intro, preview of paper, methodology!
\paragraph{Paper Overview}
In \hyperlink{sec2}{Section 2} I roughly situate the use Wittgenstein makes of this philosophical device in the context of his wider philosophy. As this requires a broader view of his philosophy, I rely on two contrasting exegetical efforts.
\hyperlink{sec3}{Section 3} is concerned with tracing Wittgenstein's \textit{Remarks on Frazer's ``Golden Bough''} and their relation to the findings of the previous section. Can Wittgenstein’s views on the proper place of anthropological description as well as his own supposed ethnographic investigations be accommodated within his wider philosophical views? Finally, \hyperlink{sec4}{Section 4} concludes and \hyperlink{sec5}{Section 5} contains the bibliography.

\section{Guidelines for Machine Learning}%\hypertarget{sec2}

\section{Application to DS}%\hypertarget{sec4}
\paragraph{•}
\paragraph{Double-nature of textual corpora}
\section{Related Work}%\hypertarget{sec3}
\section{Conclusion}

\section{References}%\hypertarget{sec5}
%\printbibliography

\end{document}